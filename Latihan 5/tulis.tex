\documentclass[conference]{IEEEtran}
\usepackage{cite}
\usepackage{amsmath,amssymb,amsfonts}
\usepackage{algorithm2e}
\usepackage{graphicx}
\usepackage{textcomp}
\usepackage{xcolor}

\title{Implementasi Algoritma Dijkstra Dalam
Menemukan Jarak Terdekat Dari Lokasi Pengguna
Ke Tanaman Yang Di Tuju}

\begin{document}

\maketitle

\begin{abstract}
    Kebun Raya Xurwodadi dengan luas area sekitar 85
    hektar ternyata kekurangan papan informasi yang menyebabkan
    pengunjung kerap kali kebingungan dalam mencari lokasi tanaman tertentu. 
    Paper ini bertujuan untuk membuat simulasi
    dari algoritma yang dapat menentukan jarak terdekat antara
    pengunjung (pengguna program) dengan lokasi tanaman yang
    dituju. Algoritma yang digunakan adalah algoritma Dijkstra
    yang beroperasi secara menyeluruh (greedy) untuk menguji
    seitap persimpangan (Vertex) dan jalan (Edge) pada Kebun
    Raya Purwodadi. Berdasarkan hasil simulasi dan pengujian,
    kompleksitas ruang dari program ini adalah O(V) karena adanya
    pembentukan array yang berisi V nodes untuk mencari heap minimum. 
    Sementara, kompleksitas waktu dari algoritma tersebut adalah O(V2).
\end{abstract}



\end{document}